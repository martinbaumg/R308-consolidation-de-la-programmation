\documentclass[12pt, a4paper]{article}
\usepackage[francais]{babel}
\usepackage{caption}
\usepackage{graphicx}
\usepackage[T1]{fontenc}
\usepackage{listings}
\usepackage{geometry}
\usepackage{minted}
\usepackage{array,multirow,makecell}
\usepackage[colorlinks=true,linkcolor=black,anchorcolor=black,citecolor=black,filecolor=black,menucolor=black,runcolor=black,urlcolor=black]{hyperref}
\setcellgapes{1pt}
\makegapedcells
\usepackage{fancyhdr}
\pagestyle{fancy}
\lhead{}
\rhead{}
\chead{}
\rfoot{\thepage}
\lfoot{Martin Baumgaertner}
\cfoot{}
\renewcommand{\footrulewidth}{0.4pt}
\renewcommand{\headrulewidth}{0.4pt}
\renewcommand{\listingscaption}{Code}
\renewcommand{\listoflistingscaption}{Table des codes}
% \usepackage{mathpazo} --> Police à utiliser lors de rapports plus sérieux

\begin{document}
\begin{titlepage}
	\newcommand{\HRule}{\rule{\linewidth}{0.5mm}} 
	\center 
	\textsc{\LARGE iut de colmar}\\[1.5cm] 
	\textsc{\Large R308}\\[0.5cm] 
	\textsc{\large Année 2022-23}\\[0.5cm]
	\HRule\\[0.75cm]
	{\huge\bfseries Consolidation de la programmation}\\[0.4cm]
	\HRule\\[1.5cm]
	\textsc{\large martin baumgaertner}\\[0.5cm] 

	\vfill\vfill\vfill
	{\large\today} 
	\vfill
\end{titlepage}
\newpage
\tableofcontents
\newpage
\section{CM 1 - 2 septembre 2022}

\subsection{Méthodes en python}
    Les arguments des méthodes peuvent être :
        \begin{itemize}
            \item Positionnels
            \item Nommé
            \item  Sous la forme de valeur non nommées
            \item  Sous la forme d'un dictionnaire
        \end{itemize}
    
\subsection{Programmation orienté object}
    Le stockage des données peut prendre plusieurs formes en pyhton comme
    par exemple la variable simple ou la liste. 

    \subsubsection{Modélisation object}
    UML : est un langage descriptif à base de pictogrammes permettant de
    représenter des schémas relationnels d'objets, de base de données, de gestion
    de processus. 

    En \texttt{UML}, les visibilités sont symbolisés par :
    \begin{itemize}
        \item + : Publique
        \item - : Protégé
        \item - Privé
    \end{itemize}
    
    \subsubsection{Les attributs}
        Les attributs sont des données simples encapsulées dans l'objet, ils
        correspondent aux qualités de l'objet

    \subsubsection{Visibilité de l'objet}
        En programmation orienté objet, on défini 3 niveaux de visiblité :
        \begin{listing}[H]
            \caption{Protection des données}
            \label{lst:settings}
            \begin{minted}{bash}
            __attribut : privé
            _attribut : protégé
            \end{minted}
        \end{listing}
   
    \subsubsection{Constructeur}
    Pour instancier un object avec des attributs d'instance, il faut ajouter la
    méthode \texttt{init(self,...)}.
    Cette méthode permet pleins de choses

    \subsection{Héritage : Python}
    Pour créer une classe héritant d'une autre classe, il suffit de la 
    déclarer enrtre parenthèse. 

        \subsubsection{Héritage : mot de clé \texttt{super}}
        Le mot clé super fiat référence à la {suoer-classe}, la classe dont on 
        hérite

    \subsection{Format de données}
        \subsubsection{JSON}
        Le JSON est un format de données permettant de stocker des données. 

    \subsection{Python et base de données}
    Pour connecter à une base de données en python, il faut installer 
    les drivers correspondant. 

    Pour mysql, un des modules à utiliser est mysql.connector etc... 

\newpage
\section{TD 1 - 6 septembre 2022}
Pour pouvoir afficher une chaîne de caractère, il faut utiliser la fonction
    
    


   


\end{document}